\begin{lstlisting}[style=peoptOutput,caption={Output generated by PEOpt when running the simple inequality example.  We explain this output in Chapter \ref{ch:Output}.},label=lst:simpleIneqOut]
Iter      f(x)      merit(x)  ||grad||  ||dx||    
1         1.40e+01  1.40e+01  6.40e+00  .         
2         7.53e+00  7.53e+00  4.88e+00  1.00e+00  
3         4.08e+00  4.08e+00  2.54e+00  7.30e-01  
4         3.69e+00  3.69e+00  6.96e-01  2.33e-01  
5         3.56e+00  3.56e+00  1.11e-01  1.01e-01  
6         3.56e+00  3.56e+00  8.16e-05  4.07e-04  
7         3.56e+00  3.56e+00  8.88e-16  2.46e-05  
8         3.56e+00  3.56e+00  8.88e-16  9.20e-08  

Iter      KryIter   KryErr    KryWhy    ared      pred      ared/pred 
1         .         .         .         .         .         .         
2         1         6.52e-01  TrstReg   6.47e+00  6.18e+00  1.05e+00  
3         2         4.48e-03  TrstReg   3.44e+00  3.26e+00  1.06e+00  
4         2         1.04e-15  RelErrSml 3.85e-01  3.54e-01  1.09e+00  
5         2         2.80e-16  RelErrSml 1.29e-01  9.59e-02  1.34e+00  
6         2         4.22e-17  RelErrSml 7.28e-04  4.08e-04  1.78e+00  
7         2         5.17e-16  RelErrSml 3.30e-05  1.71e-05  1.94e+00  
8         2         9.64e-17  RelErrSml 3.31e-07  1.72e-07  1.93e+00 

Iter      mu        mu_est    
1         1.00e-02  1.00e+00  
2         2.51e-03  2.51e-01  
3         6.88e-04  6.88e-02  
4         3.53e-04  3.53e-02  
5         3.73e-06  3.73e-04  
6         1.74e-07  1.74e-05  
7         1.75e-09  1.75e-07  
8         1.75e-09  1.75e-09  

The algorithm converged due to: RelativeGradientSmall
The optimal point is: (3.3333333399027393e-01,3.3333333399027548e-01)
\end{lstlisting}
