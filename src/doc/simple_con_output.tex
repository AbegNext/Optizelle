\begin{figure}
    \begin{lstPeopt}
Iter      f(x)      merit(x)  ||grad||  ||dx||    
1         1.40e+01  1.59e+01  3.36e+00  .         
2         4.01e+00  3.55e+00  3.97e-01  1.95e+00  
3         3.84e+00  3.70e+00  2.54e-01  8.10e-02  
4         3.57e+00  3.56e+00  1.73e-02  1.91e-01  
5         3.56e+00  3.56e+00  2.10e-13  1.08e-02 
\end{lstPeopt}
    \begin{lstPeopt}
Iter      ared      pred      ared/pred KryIter   KryErr    KryWhy    
1         .         .         .         .         .         .         
2         1.61e+01  1.20e+01  1.34e+00  1         1.73e-18  RelErrSml 
3         1.41e-01  1.50e-01  9.43e-01  1         2.00e-17  RelErrSml 
4         1.68e-01  1.40e-01  1.20e+00  1         1.59e-17  RelErrSml 
5         5.18e-04  4.34e-04  1.19e+00  1         4.15e-17  RelErrSml
\end{lstPeopt}
    \begin{lstPeopt}
Iter      ||g(x)||  mu        mu_est    
1         3.30e+00  1.00e-02  1.00e+00  
2         3.96e-01  5.70e-05  5.70e-03  
3         2.71e-01  9.00e-07  9.00e-05  
4         1.45e-02  4.78e-09  4.78e-07  
5         0.00e+00  4.77e-09  4.77e-09  
\end{lstPeopt}
    \caption{PEOpt uses this input specification to minimize the constrained problem specified in Figure \ref{fig:simpleCon}.  We explain this specification Chapter \ref{ch:Input}.} 
    \label{fig:simpleConOut}
\end{figure}
