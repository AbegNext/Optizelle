\begin{figure}
    \begin{lstPeopt}
Iter      f(x)      merit(x)  ||grad||  ||dx||    
1         2.42e+01  2.42e+01  2.33e+02  .         
2         4.73e+00  4.73e+00  4.64e+00  3.81e-01  
*         4.73e+00  4.73e+00  4.64e+00  1.00e+00  
3         4.00e+00  4.00e+00  1.74e+01  5.00e-01  
4         3.34e+00  3.34e+00  2.33e+01  5.00e-01  
5         2.58e+00  2.58e+00  8.77e+00  2.04e-01  
*         2.58e+00  2.58e+00  8.77e+00  4.91e-01  
6         2.09e+00  2.09e+00  8.48e+00  2.45e-01  
7         1.75e+00  1.75e+00  1.37e+01  3.06e-01  
8         1.20e+00  1.20e+00  2.99e+00  9.50e-02  
*         1.20e+00  1.20e+00  2.99e+00  4.15e-01  
9         9.13e-01  9.13e-01  7.24e+00  2.08e-01  
10        6.17e-01  6.17e-01  2.36e+00  1.23e-01  
11        4.64e-01  4.64e-01  8.01e+00  2.08e-01  
12        2.69e-01  2.69e-01  1.17e+00  1.23e-01  
13        1.83e-01  1.83e-01  6.37e+00  2.08e-01  
14        9.23e-02  9.23e-02  1.23e+00  1.30e-01  
15        5.33e-02  5.33e-02  5.17e+00  2.08e-01  
16        1.85e-02  1.85e-02  6.67e-01  1.03e-01  
17        9.22e-03  9.22e-03  3.68e+00  1.88e-01  
18        6.89e-04  6.89e-04  5.82e-02  4.05e-02  
19        4.05e-05  4.05e-05  2.78e-01  5.50e-02  
20        1.62e-08  1.62e-08  2.58e-04  2.83e-03  
21        2.64e-14  2.64e-14  7.13e-06  2.84e-04 
\end{lstPeopt}
    \begin{lstPeopt}
Iter      KryIter   KryErr    KryWhy    ared      pred      ared/pred 
1         .         .         .         .         .         .         
2         2         5.92e-17  RelErrSml 1.95e+01  1.94e+01  1.00e+00  
*         2         3.14e-01  TrstReg   -6.36e-01 1.54e+00  -4.14e-01 
3         2         3.54e-01  TrstReg   7.28e-01  8.14e-01  8.93e-01  
4         2         1.19e-02  TrstReg   6.67e-01  6.28e-01  1.06e+00  
5         2         3.61e-16  RelErrSml 7.53e-01  5.68e-01  1.33e+00  
*         2         1.35e-15  RelErrSml -1.48e-01 5.57e-01  -2.65e-01 
6         2         1.15e-01  TrstReg   4.92e-01  4.36e-01  1.13e+00  
7         2         2.38e-15  RelErrSml 3.38e-01  4.07e-01  8.31e-01  
8         2         3.04e-16  RelErrSml 5.58e-01  4.66e-01  1.20e+00  
*         2         3.18e-15  RelErrSml -2.09e+00 4.54e-01  -4.60e+00 
9         2         5.13e-01  TrstReg   2.83e-01  3.40e-01  8.34e-01  
10        2         7.19e-17  RelErrSml 2.96e-01  2.26e-01  1.31e+00  
11        2         1.61e-01  TrstReg   1.52e-01  1.88e-01  8.13e-01  
12        2         1.96e-17  RelErrSml 1.95e-01  1.59e-01  1.23e+00  
13        2         3.14e-01  TrstReg   8.65e-02  1.08e-01  8.02e-01  
14        2         7.50e-17  RelErrSml 9.06e-02  7.21e-02  1.26e+00  
15        2         6.56e-02  TrstReg   3.90e-02  4.57e-02  8.52e-01  
16        2         5.29e-17  RelErrSml 3.47e-02  2.89e-02  1.20e+00  
17        2         1.77e-14  RelErrSml 9.32e-03  1.30e-02  7.17e-01  
18        2         1.13e-15  RelErrSml 8.53e-03  8.14e-03  1.05e+00  
19        2         3.02e-14  RelErrSml 6.49e-04  6.60e-04  9.84e-01  
20        2         6.80e-16  RelErrSml 4.05e-05  4.03e-05  1.00e+00  
21        2         1.23e-13  RelErrSml 1.62e-08  1.62e-08  1.00e+00
\end{lstPeopt}
    \begin{lstPeopt}
The algorithm converged due to: RelativeGradientSmall
The optimal point is: (1,1)
\end{lstPeopt}
    \caption{PEOpt generates this output when running the Rosenbrock example in Figure \ref{fig:Rosen}.  We explain this output in Chapter \ref{ch:Output}.}
    \label{fig:RosenOut}
\end{figure}
